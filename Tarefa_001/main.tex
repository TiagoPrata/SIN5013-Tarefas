\documentclass[12pt]{article}
 
\usepackage[portuguese]{babel}
\usepackage[margin=1in]{geometry} 
\usepackage{amsmath,amsthm,amssymb,scrextend}
\usepackage{fancyhdr}
\pagestyle{fancy}

 
\newcommand{\N}{\mathbb{N}}
\newcommand{\Z}{\mathbb{Z}}
\newcommand{\I}{\mathbb{I}}
\newcommand{\R}{\mathbb{R}}
\newcommand{\Q}{\mathbb{Q}}
\renewcommand{\qed}{\hfill$\blacksquare$}
\let\newproof\proof
\renewenvironment{proof}{\begin{addmargin}[1em]{0em}\begin{newproof}}{\end{newproof}\end{addmargin}\qed}
% \newcommand{\expl}[1]{\text{\hfill[#1]}$}
 
\newenvironment{theorem}[2][Teorema]{\begin{trivlist}
\item[\hskip \labelsep {\bfseries #1}\hskip \labelsep {\bfseries #2.}]}{\end{trivlist}}
\newenvironment{lemma}[2][Lema]{\begin{trivlist}
\item[\hskip \labelsep {\bfseries #1}\hskip \labelsep {\bfseries #2.}]}{\end{trivlist}}
\newenvironment{problem}[2][Problema]{\begin{trivlist}
\item[\hskip \labelsep {\bfseries #1}\hskip \labelsep {\bfseries #2.}]}{\end{trivlist}}
\newenvironment{exercise}[2][Exercício]{\begin{trivlist}
\item[\hskip \labelsep {\bfseries #1}\hskip \labelsep {\bfseries #2.}]}{\end{trivlist}}
\newenvironment{reflection}[2][Reflecção]{\begin{trivlist}
\item[\hskip \labelsep {\bfseries #1}\hskip \labelsep {\bfseries #2.}]}{\end{trivlist}}
\newenvironment{proposition}[2][Proposição]{\begin{trivlist}
\item[\hskip \labelsep {\bfseries #1}\hskip \labelsep {\bfseries #2.}]}{\end{trivlist}}
\newenvironment{corollary}[2][Corolario]{\begin{trivlist}
\item[\hskip \labelsep {\bfseries #1}\hskip \labelsep {\bfseries #2.}]}{\end{trivlist}}
 
\begin{document}
 
% --------------------------------------------------------------
%                         Start here
% --------------------------------------------------------------

\lhead{EACH USP - PPgSI - SIN5013}
\rhead{Tiago Correa Prata (6907878)}
 
% \maketitle
 
\begin{problem}{1}
Prove por indução que $$2^n > 2^{n-1} + 2^{n-2} + 2^{n-3} + \cdots + 2^0 \text{, para } n \ge 1$$
\end{problem}

\begin{proof}
    Provando em $n$ \\
    \textbf{Caso base (n = 1):  } $2^1 > 2^0$ \\
    \textbf{Hipótese Indutiva (H.I.): } Verdadeiro para $n=k$, ou seja $$2^k > 2^{k-1} + 2^{k-2} + 2^{k-3} + \cdots + 2^0$$ \\
    \textbf{Tese indutiva: } Se $n = k-1$ é válido então $n = k$ é válido. Assim, se
    
    $$2^{k-1} > 2^{k-2} + 2^{k-3} + 2^{k-4} + \cdots + 2^0 \text{ é verdadeiro}$$

    Somando $2^{k-1}$ de ambos os lados, temos:

    $$2^{k-1} + 2^{k-1} \stackrel{?}{>} 2^{k-1} + 2^{k-2} + 2^{k-3} + \cdots + 2^0 \Longrightarrow 2^1.2^{k-1} \stackrel{?}{>} 2^{n-1} + 2^{n-2} + 2^{n-3} + \cdots + 2^0$$

    \begin{equation*}
        \boxed{2^{k} > 2^{k-1} + 2^{k-2} + 2^{k-3} + \cdots + 2^0}
    \end{equation*}

    $\therefore$ Pelo princípio da indução matemática, a afirmação é verdadeira para $n \in \N^*$
\end{proof}

\newpage

\begin{problem}{2}
    Prove por indução que $$1^2 + 2^2 + 3^2 + \cdots + n^2 = \frac{(2n^3 + 3n^2 + n)}{6} \text{, para todo } n \ge 1$$
\end{problem}

\begin{proof}
    Provando em $n$ \\
    \textbf{Caso base (n = 1):  } $1^2 = \frac{(2.(1)^3 + 3.(1)^2 + 1)}{6} \Longrightarrow 1=1$ (trivial) \\
    \textbf{Hipótese Indutiva (H.I): } Verdadeiro para $n=k+1$, ou seja $$1^2 + 2^2 + 3^2 + \cdots + (k+1)^2 = \frac{(2(k+1)^3 + 3(k+1)^2 + (k+1))}{6}$$ \\
    \textbf{Tese indutiva: } Se $n = k$ é válido então $n = k+1$ é válido. Sendo assim, se

    $$1^2 + 2^2 + 3^2 + \cdots + k^2 \stackrel{?}{=} \frac{(2k^3 + 3k^2 + k)}{6} \text{ é verdadeiro}$$

    Somando ${k+1}^2$ de ambos os lados, temos:

    \begin{flalign*}
        1^2 + 2^2 + 3^2 + \cdots + k^2 + (k+1)^2 &\stackrel{?}{=} \frac{2(k+1)^3 + 3(k+1)^2 + (k+1)}{6}     \\
        \\
        &\stackrel{?}{=} \frac{2k^3 + 3k^2 + k + 6k^2 + 12k + 6}{6}         \\
        \\
        &\stackrel{?}{=} \frac{2k^3 + 6k^2 + 6k + 2 + 3k^2 + 6k + 3 + k + 1}{6}         \\
        \\
        &\stackrel{?}{=} \frac{2(k^3 + 3k^2 + 3k+ 1) + 3(k^2 + 2k + 1) + (k+1)}{6}
    \end{flalign*}

    \begin{equation*}
        \boxed{1^2 + 2^2 + 3^2 + \cdots + (k+1)^2 = \frac{(2(k+1)^3 + 3(k+1)^2 + (k+1))}{6}}
    \end{equation*}

    $\therefore$ Pelo princípio da indução matemática, a afirmação é verdadeira para $n\in\N^*$
\end{proof}
 
% --------------------------------------------------------------
%     You don't have to mess with anything below this line.
% --------------------------------------------------------------
 
\end{document}